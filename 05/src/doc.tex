\documentclass[11pt, a4paper, titlepage]{article}

% Set document dimensions
\usepackage[paper=a4paper,top=2cm,left=1.8cm,right=1.8cm,bottom=2cm, includefoot]{geometry}
\usepackage{float}
\usepackage[final]{pdfpages} % inludesvg

% Czech fonts
\usepackage[T1]{fontenc}
\usepackage[utf8]{inputenc}
\usepackage[czech]{babel}
\usepackage{fancyhdr}
\usepackage{multicol}


\usepackage{background}
\setlength{\headheight}{3em} 
\newcommand{\subsectionbreak}{\clearpage}

\begin{document}
	\begin{titlepage}
    % \newgeometry{top=1in,top=2cm,left=2cm,right=2cm,bottom=2cm}

    \centering

    {\fontsize{20pt}{15pt}\bfseries
    VYSOKÉ UČENÍ TECHNICKÉ V~BRNĚ\\
    \vspace{8pt}
    Fakulta informačních technologií
    }


    \includegraphics[scale=0.7]{./assets/fit-logo.pdf}

    \vspace{22pt}

    {\Large Projekt do předmětu INI - 5. část\\}
    \vspace{16pt}
    {\LARGE \bfseries Service Review Report\\Service Improvement Plan\\Request for Change\\}
    \vspace{90pt}
    {\Large \bfseries Tým INIcializace\\}
    \vspace{50pt}
    {\Large \today}

    \vspace{90pt}
    {\Large \bfseries Autoři\\}
    \vspace{12pt}
    

    \begin{tabular}{ l c r }
        Iva Kavánková & \texttt{xkavan05} \\
        Martin Kobelka & \texttt{xkobel02} \\
        Josef Kolář & \texttt{xkolar71} \\
        Son Hai Nguyen & \texttt{xnguye16} \\
        Kateřina Šmajzrová & \texttt{xsmajz00} \\
    \end{tabular}\\

\end{titlepage}
	\pagestyle{fancy}
	\rhead{\includegraphics[height=3em]{./assets/teeth.png}}
	\lfoot{\emph{VUT FIT - INI}}
	\rfoot{\emph{Tým INIcializace}}

	\noindent Jsme společnost IZUB a poskytujeme komplexní řešení pro malé, střední i ty největší \textbf{zubařské ordinace} nejen na území České i Slovenské republiky.

	\vspace{0.5em}

	\section*{Service Review Report: Správa kartotéky pacientů}

	\noindent\makebox[\linewidth]{\rule{17.5cm}{0.4pt}}

	\vspace{0.5em}

	\noindent Následující dokument představuje zprávu typu \textbf{Service Review Report}, což je zpráva o provozování služeb. Pro tento dokument jsme si vybrali službu \textbf{Správa kartotéky pacientů}. Naše služba byla schválena v Brně dne 14. 12. 2017. Přičemž za zodpovědnou osobu byl vybrán Ing. Dominik Otevřel, pracovník ekonomického oddělení naší společnosti.

	\noindent\makebox[\linewidth]{\rule{17.5cm}{0.4pt}}

	\section{Průběh jednání jednotlivých stran}

	\begin{multicols}{2}

	Osoba zastupují společnost U Zubní Víly a.s.
	\begin{itemize}
		\item Jméno: Maxmilián Veliký
		\item Adresa: Svitavská 64, Moravská Třebová
		\item Telefon: +420 123 456 789
		\item Email: maxvel@uzubvil.cz
	\end{itemize}

	Osoba zastupují společnost U IZUB a.s.
	\begin{itemize}
		\item Jméno: Ing. Karel Joráš
		\item Adresa: Pradubická 64, Brno
		\item Telefon: +420 987 654 321
		\item Email: karel.joras@izub.com
	\end{itemize}

	\end{multicols}

	Jednání probíhalo podle podmínek uvedených ve smlouvě SLA s několika málo vyjimkami. Z dobré vůle a snahy o dosažení co nejvyšší kvality ze strany dodavatele byly zorganizovány celkem 3 mimořádné schůzky.

	\subsection{Zhodnocení kvality dodávaných služeb}

	Zhodnocení plnění kvality dojednaných služeb se oboum zúšastněným stranám jeví jako nadstandardní. Naprostá většina služeb ujednaných v SLA byla po vzájemné diskuzi ohodnocena jako kvalitně realizovaná. Veškeré výhrady jsou popsány v bodech níže.

	\subsubsection{Výpadek služby dne 28. 10. 2017}

	Dne 28. 10. 2017 došlo k plnému výpadku od 6:38 ráno do 11:17. Tento výpadek byl způsoben sérií nepravděpodobných událostí. Vše začalo naplánovanou aktualizací na termín 03:00 až 05:00, která samotná proběhla bez problémů. V 5:55 došlo zasměstnacem odběratele Bc. Janem Nešikou k neoprávněnému a neodbornému zásahu do IT infastruktury, kvůli kterému došlo k úplně výpadku záložních systémů. V 6:03 na místo datacentra vyrazila technická podpora společnosti IZUB.

	Technici místo po příjezdu v 6:29 zajistili a začali provádět úkony pro obnovení těch záložních systémů. V 6:38 ovšem došlo k výpadku napájení pro celé datacentrum, což vyřadilo z běhu i hlavné systémy pro provoz Služby. Tento výpadek trval až do 10:55 a následně byl u obou systémů v 11:17 obnoven plný provoz.

	Dle SLA neplyne z tohoto incidentu žádný právní postih pro dodavatele. Dodavatel doporučuje pracovně právní postih pro zaměstnance Jana Nešiku a apeluje na kontrolu zodpovědnosti a kompetence zaměstnanců ve společnosti odběratele.

	\subsubsection{Navýšení limitů dne 1. 12. 2017}

	Dle nových smluvních podmínek dodavatele IZUB platných od 1. 12. 2017 došlo ke zvojnásobení limitu u všech balíčků - pro společnost odběratele tedy nyní platí balíček STANDARD s limitem 1024 evidovaných pacientů. Tato skutečnost je platná od 1. 12. a byla dodavatelem automaticky zaintegrována do Služby dne 23. 11. zásahem techniků.

	\subsection{Plánovaná vylepšení}

	Odběratel již na minulých setkáních vyjádřil zájem o integraci bezpečnostních aspektů definovaných ve službě IT infrastruktura. Na 7. 1. 2018 je naplánovaná schůzka IT analytiků z obou zůčastněných stran a na této schůzce bude konzultován plán k rozšíření poskytovaných služeb. Finálně bude odsouhlaseno do 28. 2. 2018. 


	\section*{Improvement plan: Správa kartotéky pacientů}

	\vspace{0.5em}

	\noindent\makebox[\linewidth]{\rule{17.5cm}{0.4pt}}

	\vspace{0.5em}

	\noindent Následující část dokumentu popisuje \textbf{Improvement Plan} ke službě \textbf{Správa kartotéky pacientů}. Přičemž za zodpovědnou osobu byl vybrán Bc. Vlastimil Mohavica, pracovník ekonomického oddělení naší společnosti.

	\noindent\makebox[\linewidth]{\rule{17.5cm}{0.4pt}}

	\subsection{Zvýšení ochrany vůči výpadkům}

	Počátek realizace naplánován na \textbf{31.12.2017}, plánované ukončení realizace \textbf{8.3.2018}.

	\subsubsection{Návrh na řešení}

	Následující návrhy řešení jsou seřazeny sestupně podle priority.

	\begin{enumerate}
		\item Přidání záložního zdroje v podobě naftového generátoru. Zodpovědná osoba Ing. Martin Martínek.
		\item Zamezení přístupu nežádoucích osob ke klíčovým prvkům IT infrastruktury. Zodpovědná osoba Ing. Martin Martínek
		\item Automatický systém obnovy dat v případě výpadku. Zodpovědná osoba Ing. Martin Martínek
	\end{enumerate}

	Službu lze nazvat kritickou, z toho důvodu je zajišťění téměř 100\% dostupnosti a nulových výpadků samozdřejmostí. Změna vychází z požadavků zákazníka na dostupný systém za všech okolností a nemožnosti zaměstnance způsobit svoji nevědomostí problémy s konektivitou.

	\subsection{Zvýšení maximálních limitů u všech služeb}

	Počátek realizace naplánován na \textbf{31.12.2018}, plánované ukončení realizace \textbf{8.3.2019}.

	\subsubsection{Návrh na řešení}

	Následující návrhy řešení jsou seřazeny sestupně podle priority.

	\begin{enumerate}
		\item Monitorování přechodu na objemější databáze na zákaznických strojích poskytovaných v rámci IT infrastruktury. Zodpovědná osoba Ing. Lukáš Lukášek
	 	\item Vyšetřenů problému s nedostatečnou kapacitou klientských strojů. Zodpovědná osoba Ing. Lukáš Lukášek
	 	\item Monitoring spokojenosti klientů s dvojnásobným navýšením kapacitního stropu. Zodpovědná osoba Ing. Lukáš Lukášek
	\end{enumerate}

	Velké množství našich klientů brzy přesáhlo maximálního kapacitního stropu poskytovaných našimi limity a byli nuceni řešit toto individuálně. Zdvojnásobení limitů eliminuje tyto případy a sníží nutnou administrativitu spojenou s poskytováním výše uvedené služby.

	\subsection{Fyzické zabezpečení datacentra}

	Počátek realizace naplánován na \textbf{31.12.2019}, plánované ukončení realizace \textbf{8.3.2020}.

	\subsubsection{Návrh na řešení}

	Následující návrhy řešení jsou seřazeny sestupně podle priority.

	\begin{enumerate}
		\item Instalace kameravého do prostorů se servery v našem datacentru. Zodpovědná osoba Bc. Pavel Pavlásek
		\item Přístup do prostor se servery pouze přes několikanásobnou autentizace s využitím biometrických údajů. Zodpovědná osoba Bc. Pavel Pavlásek
		\item Zvýšení protipožární ochrany v prostorách datacentra. Zodpovědná osoba Bc. Zodpovědná osoba Bc. Pavel Pavlásek
	\end{enumerate}

	Velké množství našich klientů tvoří již i velké společnost s několika tisíci pacienty a vysokým renomé. Tyto společnosti vyjádřily obavu vůči úniku dat či ztrátě dat. Pro tyto vysoce náročné klienty jsme přichystali možnost extrémně zabezpečeného datacentra.

	\subsection{Monitorování a report}

	\section*{Request for Change: Správa kartotéky pacientů}

	\vspace{0.5em}

	\noindent\makebox[\linewidth]{\rule{17.5cm}{0.4pt}}

	\vspace{0.5em}

	\noindent Následující část dokumentu představuje \textbf{Request for Change}, což je žádost o změnu, tj. uvádí co je třeba splnit, ale nevynechává, jak by měla být změna provedena.

	\noindent\makebox[\linewidth]{\rule{17.5cm}{0.4pt}}

	\begin{enumerate}
		\item Název změny: Zvýšení maximálních limitů u všech služeb
		\item Unikátní ID: 5465469-XD
		\item Priorita:	Střední
		\item Deadline: 8.3.2019
	\end{enumerate}

	\subsection{Vlastnictví a iniciace}

	Vlastník a iniciátor: \textbf{Ing. Radek Radeček}

	\subsection{Související dokumenty}
	\begin{itemize}
		\item Email od dynamicky se rozvíjející zahraniční společnosti Tear a.s se zájmem proniknout na český trh ze dne 14.3.2016.
		\item Problem report C545546546DH-86
		\item Analýza klientské základny ze dne 21.12.2012
	\end{itemize}

	\subsection{Detailní rozpis změny}

	V prvním kroku zjistíme, zda klientovy stroje vykazují dostatečný potenciální výkon a kapacitu potřebnou pro uložení dvojnásobného množství dat. V druhém kroku dojde k podepsání dodatku smlouvy. Ve třetím kroku dojde se souhlasem klienta k samotnému zvýšení limitů Služby.

	\subsection{Dopady změny}

	Neočekávají se žádné negativní dopady změny na bussines služby klienta. Je očekávána pozitivní změna ve smyslu možnosti rozšíření bussines služeb odběratele služby. 


\end{document}
